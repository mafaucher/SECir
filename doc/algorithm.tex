%
% Template Author: Marc-André Faucher
% Last update: 10-10-10
%
% 
%%%%%%%%%%%%%%
% REMINDERS: %
%%%%%%%%%%%%%%
%
% Illegal Characters: # $ % & ~ _ ^ \ { }
%
% Enumerate: \begin{enumerate}[(a)] [1.] [CHAPTER 1 - ]
%		\item 1st entry
%		\item 2nd entry ...
%	     \end{enumerate}
%
% Include Graphics: \begin{center} \includegraphics[scale=0.5]{fig01.png} \end{center}
%
% Force line break: \\
% 	Additional: \item \hspace*{\fill} \\ 
%
% Force page break: \pagebreak
% 
% 

\documentclass[10pt]{article}
\usepackage[margin=1in]{geometry}
\usepackage[pdftex]{graphicx}
\usepackage{amssymb,amsmath,enumerate,fancyhdr,listings}

\pagestyle{fancy}
\fancyhf{}
\lhead{Document Extraction Algorithm}
\rhead{May~19,~2012}


%%%%%%%%%%%%%%%%%%%%%%%%%
% MAIN DOCUMENT CONTENT %
%%%%%%%%%%%%%%%%%%%%%%%%%

\begin{document}

\begin{enumerate}[A - ]
    \item   {\bf Parsing Header Information:}
        
        \begin{enumerate}[1.]
            
            \item   For each document, get the header information within the SEC-XML tags
                    \texttt{<SEC-HEADER>} and \texttt{</SEC-HEADER>}.
        
            \item   From the header, retrieve the following information:
                \begin{enumerate}[i)]
                    \item   \texttt{COMPANY CONFORMED NAME};
                    \item   \texttt{CENTRAL INDEX KEY};
                    \item   \texttt{FORM TYPE};
                    \item   \texttt{CONFORMED PERIOD OF REPORT} (\emph{optional});
                    \item   \texttt{FILED AS OF DATE};
                \end{enumerate}
            
            \item   The document naming scheme is: \texttt{<CENTRAL INDEX KEY>-<FILED AS OF DATE>.txt}.
            
            \item   If an item is successfully extracted from this document, write an
                    entry into the comma-separated-value (CSV) file.

            \item   If not, the unparsed document is added to the log file.
        
        \end{enumerate}

    \item   {\bf Parsing Text Information:}

        \begin{enumerate}[1.]

            \item   Get the document text within the first SEC-XML tags
                    \texttt{<TEXT>} and \texttt{</TEXT>}.

            \item   If the document contains HTML data, we need to remove any HTML tags.
                    Before doing so, add line breaks before any \texttt{<P>} HTML tag to simplify
                    paragraph detection in the next phase. Next ignore any HTML tags, and
                    convert HTML entities in the form \texttt{\&\#...;} to their unicode value.
                    Additionally, convert non-breaking spaces (U+00A0) to regular spaces,
                    since non-breaking spaces are not considered whitespaces by the regular
                    expressions.

            \item   Attempt to find an ``Item 5" or, if none is found, an ``Item 9":
                \begin{enumerate}[i)]
                    \item   Match the text between the desired item title and the next item
                            using tolerant regular expressions. More specifically, we identify
                            a line starting with \texttt{ITEM 5 Operating} and a line
                            starting with \texttt{ITEM 6}. For ``Item 9", we identify a line
                            starting with \texttt{ITEM 9 Management's Discussion} and a line
                            starting with \texttt{ITEM 10}. In either case, we tolerate different
                            capitalization, additional whitespaces, and formatting characters
                            after the item number.
                            \footnote{The full regular expressions can be seen at:
                            \texttt{https://github.com/mafaucher/SECir/blob/master/src/parser.py}.}
                    \item   If the matched text still contains the desired item title we ignore
                            everything before the last item title which contains at least 100 words.
                            This is mainly to avoid matching the table of content entry for the
                            desired item.
                \end{enumerate}

            \item   If we were unable to match anything, or the matched text contains fewer than
                    100 words, then no text is extracted from this document.

            \item   If the item is successfully extracted, we remove additional markup:
                \begin{enumerate}[i)]
                    \item   A page number followed by a \texttt{<PAGE>} tag and any whitespace
                            before the page number, and after the tag. This is important for
                            paragraph detection in the next phase.
                    \item   Any table within \texttt{<TABLE>} and \texttt{</TABLE>} tags.
                    \item   Any other remaining tags contained in angle brackets \texttt{<...>}.
                \end{enumerate}

        \end{enumerate}

    \item   {\bf Extracting Paragraphs}

        The data extracted from the last phase may still contain subtitles, page
        numbers, tables, and other content we wish to ignore. This phase cleans up
        the text by removing as much of this content as possible.

        \begin{enumerate}[1.]

            \item   First, we split the text into paragraphs using empty lines which
                    contain only whitespace to determine the seperation between one
                    paragraph and the next.

            \item   Next we filter out the paragraphs which do not end with a punctuation:
                    `.', `,', `:', `;', `!', `?'.

        \end{enumerate}

\end{enumerate}

\end{document}
